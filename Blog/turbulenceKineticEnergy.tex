There are numerous metrics that can be used to investigate a turbulent flow. Some are considered to be general turbulence descriptors (i.e., turbulence kinetic energy), whereas others provide only a detailed piece of information on a specific turbulent property of the flow (i.e., the Reynolds shear stresses). Some of the most commonly appearing examples includes: turbulence kinetic energy (often abbreviated k or TKE), root-mean square (RMS), turbulence intensity (TI) and Reynolds shear stresses (RSS). There are also many ways these metrics can be presented to help better analyze turbulence data. Each metric has its 'use' to help draw conclusions on the turbulent structure of the flow. Because of the inherent strengths and weaknesses of each metric, discussing them in combination is often needed to form a complete picture of a turbulent flow field, or at least enough of one to support the conclusions of a study. Having a good grasp of the 'meaning' of each metric is definitely useful in such a task. 


The vast number of turbulence metrics available and the variety of ways in which they can be presented can make the study of turbulence overwhelming. The goal of this series of blog posts is to provide high-level, intuitive explanations of the most important turbulence metrics. The posts are supported by code examples (in Python) working with turbulent velocity measurements taken from laboratory and field experiments. The intent is not to provide a detailed derivation of each metric (I provide references to those for those interested), but instead round-out the reader's intuitive understanding of the turbulence descriptors and to help answer questions such as "What does turbulence intensity tell me?" or, "What is the best way to demonstrate that my flow contains isotropic or anisotropic turbulence?". 
